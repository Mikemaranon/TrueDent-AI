\documentclass[a4paper,12pt]{article}

\usepackage[utf8]{inputenc}
\usepackage[spanish]{babel}
\usepackage[T1]{fontenc}
\usepackage{graphicx}
\usepackage{amsmath, amssymb}
\usepackage{geometry}
\usepackage{hyperref}
\usepackage{bookmark}
\usepackage{enumitem}
\geometry{margin=2.5cm}

\title{Documentación Técnica de TrueDent AI}
\author{
  Mike Marañón \\ 
  Javier Pérez \\ 
  Proyecto de Inteligencia Artificial en Odontología
}

\date{\today}

\begin{document}

\maketitle
\tableofcontents


\section*{I. Contexto del Proyecto}

El dentista Ángel Cidad se puso en contacto con nosotros a través de Jose Antonio Ureta Santacreu (JAUS) para poder realizar este proyecto. Su objetivo era desarrollar un producto basado en Inteligencia Artificial que ayudara a los dentistas de su clínica a detectar patologías a través del análisis de pantomografías dentales.

A pesar de que los dentistas son profesionales altamente capacitados, el mundo de la salud es delicado: no detectar una patología a tiempo puede derivar en una intervención quirúrgica. La existencia de TrueDent busca acelerar y facilitar ese proceso de detección, sobre todo en los casos donde los síntomas no son fácilmente visibles.

Las tecnologías de convolución de imágenes son especialmente útiles en este campo, ya que el análisis de unos pocos píxeles puede marcar la diferencia entre un diagnóstico certero y un falso negativo.

\section*{II. Introducción}

TrueDent pretende ser una solución orientada a profesionales del ámbito odontológico. Con la ayuda de Ángel Cidad y el apoyo institucional de Tajamar, hemos desarrollado este MVP que reconoce caries en una pantomografía dental, evaluando cada diente de manera independiente.

El desarrollo se estructura en cuatro fases, cada una enfocada en un modelo distinto, lo que permite modular el sistema y adaptarlo a diferentes contextos clínicos.

\subsection*{II-1. Fases del Proyecto}

\subsection*{Fase 0: Preparación previa y arquitectura}
Antes de entrenar cualquier modelo, diseñamos la arquitectura general y establecimos el entorno de trabajo. Esta fase incluye la organización del código, herramientas, estructura de carpetas y pipeline general.

\subsection*{Fase 1: Detección de dientes}
Entrenamos un modelo basado en YOLOv8m para detectar y extraer imágenes individuales de cada diente desde la pantomografía.

\subsection*{Fase 2: Clasificación y segmentación}
Incluye dos modelos:

\begin{itemize}
    \item \textbf{Clasificador CNN:} determina si un diente es sano o no.
    \item \textbf{Segmentador CNN (futuro):} detectará visualmente las áreas dañadas (por ahora no implementado en el MVP).
\end{itemize}

\subsection*{Fase 3: Generación de diagnóstico}
Un modelo generativo produce un informe textual que explica las patologías detectadas y los motivos detrás de su clasificación.

\subsection*{Resultado esperado}

El producto final es una aplicación web donde el usuario sube una pantomografía y recibe:

\begin{itemize}
    \item Una imagen procesada con cada diente identificado.
    \item Un informe textual explicando los hallazgos relevantes.
\end{itemize}

\section*{III. Análisis de la realidad}

Antes de comenzar, identificamos una serie de desafíos. El equipo está compuesto por expertos en IA y Big Data, pero no en odontología, por lo que recurrimos a la colaboración de Ángel Cidad para validar muchos resultados.

\subsection*{Conocimientos adquiridos}

Para trabajar de forma autónoma, investigamos:

\begin{itemize}
    \item La nomenclatura dental.
    \item Características visuales comunes de una caries.
    \item Diferencias entre caries y otras patologías.
\end{itemize}

\subsection*{Problemas con los datos}

Detectamos limitaciones clave:

\begin{itemize}
    \item Etiquetas en formatos normalizados (como YOLO) dificultan la validación visual.
    \item Muchos datasets no se ajustan al contexto clínico deseado.
    \item No contamos con tiempo suficiente ni recursos para revisar la integridad completa de los datos.
\end{itemize}

\textbf{Conclusión:} Aún con datos perfectos, ningún modelo es infalible. Por eso, el sistema debe permitir al usuario validar visualmente y no depender ciegamente del modelo.

\section{Arquitectura y tecnologías del sistema}

TrueDent se compone de cuatro modelos IA + una aplicación web como interfaz de usuario.

\subsection*{Preguntas planteadas}

\begin{itemize}
    \item ¿Qué precisión se espera alcanzar?
    \item ¿Qué tipo y calidad de datos se tienen disponibles?
    \item ¿Cuál es la complejidad esperada de la app?
    \item ¿Quién será el usuario final?
\end{itemize}

Originalmente se quería detectar todo tipo de patologías. Por cuestiones de tiempo, el MVP sólo diferencia entre dientes sanos y no sanos.

\subsection*{Requisitos de datasets}

\begin{enumerate}
    \item Radiografías panorámicas con máscaras de cada diente.
    \item Imágenes individuales de dientes, con etiqueta binaria: \texttt{sano} / \texttt{no sano}.
\end{enumerate}

\section{Resumen de modelos}

\begin{center}
\begin{tabular}{|l|l|p{8cm}|}
\hline
\textbf{Modelo} & \textbf{Tipo} & \textbf{Función principal} \\
\hline
Detector de dientes & CNN (YOLOv8m) & Extrae automáticamente cada diente a partir de una pantomografía dental. \\
\hline
Clasificador & CNN & Determina si un diente es sano o no. \\
\hline
Segmentador (futuro) & CNN & Localiza regiones afectadas dentro del diente para analizar la patología. \\
\hline
Generador de diagnóstico & Modelo generativo de texto & Redacta un informe textual con la posible patología detectada. \\
\hline
\end{tabular}
\end{center}

\end{document}